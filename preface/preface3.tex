\markboth{最后的话}{最后的话} \vspace*{0.0cm}
\thispagestyle{empty}
%\vspace*{2.2cm}
\centerline{\hei{\Large 最~后~的~话}}\vspace{2cm}

Yeats【W. B. Yeats,威廉.巴特勒.叶慈】有句名言:”教育不是注满一桶水,而是点燃一把火。”他既是对的但同时也是错的\footnote{如果他确实说了这句话;如许多名人名言一样,这句话的历史也不明}。你必须”注满那桶水”,这些笔记正好在此帮助你的学业;毕竟,当你去Google面试,他们问你一个关于如何使用信号量的刁钻问题,最好还是真的知道信号量是什么,对吧!

但是叶慈的重点明显是:教育真正的关键在于使你能对某样事务产生兴趣,去学习一些自己觉得更重要的东西,而不仅仅是某些课程中为了得高分而需消化的东西。Remzi的父亲曾经说过:”走出课堂学习”。

我们撰写了这些笔记来激发你对操作系统的兴趣,从而去围绕这个话题自己做更多的阅读,去跟你的教授讨论这个领域正在进行的有意思的研究,甚至参与到这个研究中来。这是一个很伟大的领域,充满了精彩美妙思想,它们以深刻而重要的方式塑造了计算机历史。我们知道虽然这把火不能燃烧你们所有人,至少我们希望能够燃烧你们一部分人,甚至几个人。因为一旦这把火被点燃了,那就是你们真正可以做一些伟大事情的时候。教育过程的真正关键是:行动,去研究新的有趣的话题,去学习,去成长,还有最重要的是去寻找能够点燃你的事情。


\vspace{1cm}

\hfill JAndrea and Remzi\hspace{0.2em}

\hfill Married couple \hspace{0.2em}

\hfill Professors of Computer Science at the University of Wisconsin\hspace{0.2em}

\hfill Chief Lighters of Fires, hopefully\footnote{如果这个听起来我们如纵火犯一样在承认一些历史的话,那你可能抓错重点了。也许,如果这听起来俗气的话,那好,因为他就是这么俗气,但你门不得不为此原谅我们。}\hspace{0.2em}
