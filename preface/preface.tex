\markboth{致教育工作者}{致教育工作者} \vspace*{0.0cm}
\thispagestyle{empty}
%\vspace*{2.2cm}
\centerline{\hei{\Large 致教育工作者}}\vspace{2cm}

如果你是希望使用本书的讲师或教授,随时欢迎。可能你已经注意到,本书是免费的,且可以从下面网页中获取:\\
http://www.ostep.org\\
你也可以从lulu.com网站购买纸质版。请在上述网页上查询。

本书的推荐引用格式(截止至目前)如下:\\
~~~~Operating Systems: Three Easy Pieces\\
~~~~Remzi H. Arpaci-Dusseau and Andrea C. Arpaci-Dusseau\\
~~~~Arpaci-Dusseau Books, Inc.\\
~~~~May, 2014 (Version 0.8)\\
~~~~http://www.ostep.org\\

这个课程分成一个学期15周比较好,这样可以覆盖书中的大部分话题,也能达到比较适中的深度。要是把本课程压缩到10周的话,可能需要每个部分都舍弃一些细节。还有一些关于虚拟机监控器的章节,通常压缩到虚拟化的结尾部分或者作为一个aside放在最后。

许多操作系统的书都会将并发性部分放在前面,本书不太一样,将其放在了虚拟化部分之后,这样在此之前可以对CPU和内存的虚拟化有一定理解。从我们15年教授此课程的经验来看,如果学生们没有理解地址空间是什么,进程是什么,或者上下文切换会在任意时间发生,他们都会有个困惑期——不知道并发性问题是如何引起的,为什么要试图解决这个问题。然而,一旦他们理解了虚拟化中的那些概念,当介绍线程和由线程引起的问题时会变得很简单,起码会简单些。

你可能注意到本书没有与之对应的幻灯片。主要原因是我们相信这个过时的教学方法:粉笔和黑板。因此,当我们在讲授这门课程时,我们脑子里带几个主要的思想和一些例子到课堂上,用板书的形式呈现给学生;讲义以及现场编写演示代码也是很有用的。在我们的经验里,使用太多的幻灯片会使得学生心思不在课堂上,因为他们知道反正资料就在这里,可以以后慢慢消化;板书的话会使得课堂像一个现场观看体验,因此会更有互动性、动态性,也会使学生们更好的享受你的课。

如果你想要一份我们上课的笔记的话,可以给我们发邮件。我们已经共享给了全世界很多老师。

最后一个请求:如果你使用网页上的免费章节,请仅仅链接到他们,而不是拷贝到本地。这会帮助我们追踪这些章节的使用情况(过去几年里已经有超过一百万的章节下载量),也可以确保学生们获得的是最新最好的版本。
