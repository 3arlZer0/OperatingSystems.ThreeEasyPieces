\chapter{本书的一段对话}
\thispagestyle{empty}


\textbf{教授:}欢迎来到这本书!它叫作\textbf{操作系统的三条简洁之道},我在这里会教你们一些你必须了解的操作系统知识。我是「教授」;你们呢?\newline
\textbf{学生:}你好,教授!我是「学生」,你应该猜到了。我在这里准备向您学习。\newline
\textbf{教授:}听起来不错。有什么问题不?\newline
\textbf{学生:}当然,为什么这本书叫作『三条简洁之道』呢?\newline
\textbf{教授:}很简单,你看,这是物理学家理查德·费曼的课程……\newline
\textbf{学生:}哦,是那位写了『别闹了,费曼先生』的那个人吗?好书,这本书是不是也会像那本书一样幽默?\newline
\textbf{教授:}额,不是的。那本书的确很不错,我很高兴你读过那本书。这本书更像他在物理学上的笔记。一些基础被定义为归纳为『六条简洁之道』。他是讨论的物理方面的知识;而我们是讨论以计算机为主题的『三条简洁之道』。差不多,操作系统正好是物理学一半的难度。\newline
\textbf{学生:}那真好,我也喜欢物理。是哪些简洁之道呢?\newline
\textbf{教授:}我们将会学到计算机的三个关键点:『\textbf{虚拟化,并发,持久化}』。在学习这些知识点的时候,我们会了解整个操作系统是如何运作的,包括 CPU 如何决定下一个运行什么程序。如何将内存载入到虚拟内存中。虚拟机监控如何运作,怎样管理硬盘信息,以及一些如何构建部分掉线但仍然可以工作的分布式系统。这是这些东西的主要顺序。\newline
\textbf{学生:}我对您说的这些还没什么概念。\newline
\textbf{教授:}很好,那说明你上对了课了。\newline
\textbf{学生:}我还有另外一个问题:学这些东西最好的方法是什么?\newline
\textbf{教授:}问的好!每个人都需要了解这些,当然,我会这样做:去上课,听教授介绍材料。然后到周末阅读自己的笔记,来帮助你的大脑更深入的理解。一段时间后(提示:在考试之前),再阅读这些笔记来牢固知识。当然,你的教授肯定会布置一些作业和项目,你必须完成;尤其是做项目的时候你为了解决问题而写的代码,是实践笔记中的知识最好的方式。孔子曾经说过……\newline
\textbf{学生:}哦,我知道!『不闻不若闻之,闻之不若见之,见之不若知之,知之不若行之。学至于行之而止矣。』是这句或者类似的吧。\newline
\textbf{教授:}(惊讶)你怎么知道我想说的呢?\newline
\textbf{学生:}因为接下去就像这句啊。而且我非常崇拜孔子,更加崇拜荀子。此句应该是荀子所言。\newline
\textbf{教授:}(震惊)好吧,我想我们要一起好好的努力了。真正的加油。\newline
\textbf{学生:}教授——如果可以的话,我还剩一个问题。这段对话是做什么用的啊?我的意思是,这不应该是一本书吗?为什么不直接介绍实质性的东西呢?\newline
\textbf{教授:}嗯,好问题,好问题!我认为和你自己单独聊这些和思考问题更有用,这段对话就是为了这个。所以你和我要一起工作来完成这个复杂的点子。你准备好了吗?\newline
\textbf{学生:}那我们必须思考?好吧,我准备好了。我的意思是,有一些其他的我能做的吗?貌似我在这本书之外没有太多的出场机会。\newline
\textbf{教授:}不幸的是,我也没有。好了,我们开始工作吧。